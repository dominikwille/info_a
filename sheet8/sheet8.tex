\documentclass[a4paper, parskip=half]{scrartcl}
\usepackage{amsmath}
\usepackage[utf8]{inputenc}

\begin{document}
\textbf{\LARGE Sarah Zewge und Dominik Wille - Freitags}
  \section{Rekursion}
\begin{verbatim}
g :: Int -> Int
g n = g_ n 3 [2, 1, 0]

g_ :: Int -> Int -> [Int] -> Int
g_ n k l
  | (n < 3)   = n
  | (n == k)  = g__ l
  | otherwise = g_ n (k + 1) ([g__ l] ++ l) 

g__ :: [Int] -> Int
g__ (g1:g2:g3:l) = g1 + 2 * g2 - g3
\end{verbatim}
  \section{Tupel}
\subsection*{a)}
\begin{verbatim}
pointOnLine (a, b) (m, n) = (b == m * a + n)
pointOverLine (a, b) (m, n) = (b > m * a + n)
\end{verbatim}
\subsection*{b)}
\begin{verbatim}
lineThrough (a, b) (c, d)
  | (a == c)  = error "That is not possible (x1 = x2)"
  | (a < c)   = (((d - b) / (c - a)), (-((d - b) / (c - a)) * a + b))
  | otherwise = lineThrough (c, d) (a, b)
\end{verbatim}
\subsection*{c)}
\begin{verbatim}
crossing (m, n) (o, p)
  | (m == o)  =  error "That is not possible (Lines have the same slope)"
  | otherwise = (((p - n) / (m - o)), ((p * m - n * o) / (m - o)))
\end{verbatim}
\subsection*{d)}
\begin{verbatim}
parallelThrough (m, n) (a, b) = (m, (-m * a + b))
\end{verbatim}
  \section{Binominialkoeffitienten}
  \subsection*{a)}
\begin{verbatim}
factorial n 
  | (n <= 1)  = 1
  | otherwise = n * factorial (n - 1)  

biomDef n k = (factorial n) / (factorial k) / (factorial (n - k))
\end{verbatim}
  \subsection*{b)}
  \begin{align}
    \binom{n - 1}{k} + \binom{n - 1}{k - 1} &= \frac{(n - 1)!}{k!(n - k - 1)!} + \frac{(n-1)!}{(k-1)!(n-k)!} \\
    &= \frac{(n-1)!(n-k)}{k!(n-k)!} + \frac{(n-1)!k}{k!(n-k)!} \\
    &= \frac{n(n-1)!}{k!(n-k)!} \\
    &= \frac{n!}{(n-k)!} \\
    &= \binom{n}{k} \quad \textbf{q.e.d.}
  \end{align}
  \subsection*{c)}
\begin{verbatim}
biomRec n k 
  | (n == k || k == 0) = 1
  | otherwise          = (biomRec (n - 1) k) + biomRec (n - 1) (k - 1)    
\end{verbatim}
\subsection*{d)}
\begin{verbatim}
countCalls n k = countCalls_ n k 0
countCalls_ n k a
  | (n == k || k == 0) = a + 1
  | otherwise = (countCalls_ (n-1) k (a+1)) + countCalls_ (n-1) (k-1) (a+1) 
\end{verbatim}
Die Konstanten $c, d$ können beispielsweise beide $= 1$ gewählt werden. 
\subsection*{e)}
Um den Beweis möglichst einfach zu machen werde ich tatsächlich nur die geraden Therme von $n$ betrachten. Es sei dabei $2k= n$. Zeigen möchten wir, dass \verb+binomRec 2k k+ midestens $2^k$ Aufrufe hat.\\ \\
\textbf{Induktionsanfang:} \\
\verb+biomRec 0 0+ hat einen Aufruf \\\\
\textbf{Induktionsschritt:}\\
\verb§biomRec 2(k + 1) (k + 1)§ Macht auf jeden Fall zwei Aufrufe auf \verb§biomRec 2k k§ und hat daher midestens doppelt so viele Aufrufe wie  \verb§biomRec 2k k§. Das heißt die Aufrufe steigen midestens exponentiell mit $k$. \textbf{q.e.d.}\ \\ \\

\end{document}
