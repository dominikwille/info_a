\input{../abv/lib/my}

\newcommand{\myTitle}{Mitschrieb - Informatik A}
\newcommand{\myAuthor}{Dominik Wille & Sarah Zewge}
\newcommand{\myDate}{Wintersemester 2013/2014}
\newcommand{\myTitleImage}{} %This Image is GNU GPL licenced.
\newcommand{\myTitleLeft}{%
   Freie Universität Berlin\\
   Fachbereich für Informatik\\
   Informatik A%
}
\newcommand{\myTitleRight}{%
  Dozent: \\
  Klaus Kriegel
}

\begin{document}
\myTitlepage
\section{Syntax und Semantik}
\subsection{Repräsentation und Bedeutung von Information} \\
\textbf{Definition Informatik (Broy):}\\
\\
 Wissenschaft Technik und Anwendung der maschinellen Verarbeitung, Speicherung und Übertragung von Information.

\textbf{Information :}
\begin{itemize}
\item \textbf{äußere Gestalt} Repräsentation/Darstellung muss bestimmten Regeln genügen (\textbf{Syntax})
\item \textbf{Bedeutung} Kern der Information (\textbf{Semantik})
\end{itemize}

\textbf{Informationssystem :}\\
\\
Besteht aus einer Menge von zulässigen Repräsentationen $R$ und einer Menge von abstrakten Informationen $A$, sowie einer Interpretation

\begin{equation}
I: R \rightarrow A
\end{equation}

\subsection{Natürliche Zahlen}
\textbf{Peano - Axiome}
\begin{enumerate}
\item $0$ ist eine natürliche Zahl.
\item Für jede natürliche Zahl $n$ gibt es einen eindeutigen Nachfolger $s(n)$, der auch eine natürliche Zahl ist.
\item verschiedene natürliche Zahlen haben verschiedene Nachfolger.
\item $0$ ist keine Nachfolger einer natürlichen Zahl.
\item Die Menge $\mathbb{N}$ ist die kleinste\footnote{im Sinne der Inklusion} Menge, die 1-4 erfüllt.
\end{enumerate}

\textbf{Axiome}
\begin{itemize}
\item \textbf{Rekursion} Definitin von Operationen/Funktionen auf $\mathbb{N}$.
\item \textbf{Induktion} Beweise, dass Aussagen auf allen $n \mathbb{N}$ gelten.
\end{itemize}
Rekursion besteht aus:
\begin{enumerate}
\item \textbf{Anker} Def von $f(0)$ (oder $f(0)$, $f(1)$,...,$f(n)$)
\item \textbf{Rekusionsvorschrift} Definition von $f(n+1)$ aus $f(n)$
\end{document}
